\documentclass[conference]{IEEEtran}
\IEEEoverridecommandlockouts
% The preceding line is only needed to identify funding in the first footnote. If that is unneeded, please comment it out.

\usepackage{cite}
\usepackage{amsmath,amssymb,amsfonts}
\usepackage{algorithmic}
\usepackage[printonlyused]{acronym}
\usepackage{graphicx}
\usepackage{textcomp}
%\usepackage[backend=bibtex,style=IEEEtr]{biblatex}


\def\BibTeX{{\rm B\kern-.05em{\sc i\kern-.025em b}\kern-.08em
    T\kern-.1667em\lower.7ex\hbox{E}\kern-.125emX}}
\begin{document}

\title{Activity Recognition with mobile devices\\
}

\author{\IEEEauthorblockN{1\textsuperscript{st} Simon Angerbauer}
\IEEEauthorblockA{\textit{Mobile Computing} \\
\textit{University Of Applied Sciences Upper Austria}\\
Hagenberg, Austria \\
simon.angerbauer@students.fh-hagenberg.at}
\and
\IEEEauthorblockN{2\textsuperscript{nd} Paul Schmutz}
\IEEEauthorblockA{\textit{Mobile Computing} \\
\textit{University Of Applied Sciences Upper Austria}\\
Hagenberg, Austria \\
paul.schmutz@students.fh-hagenberg.at}
\and
\IEEEauthorblockN{3\textsuperscript{rd} Roman Socovka}
\IEEEauthorblockA{\textit{Mobile Computing} \\
\textit{University Of Applied Sciences Upper Austria}\\
Hagenberg, Austria \\
roman.socovka@students.fh-hagenberg.at}
}

\maketitle

\begin{abstract}
This document is a model and instructions for \LaTeX.
This and the IEEEtran.cls file define the components of your paper [title, text, heads, etc.]. *CRITICAL: Do Not Use Symbols, Special Characters, Footnotes, 
or Math in Paper Title or Abstract.
\\
\\
\\
\\
\\
\\
\\
\\
\\
\\
\\
\\
\\
\\
\\
\\
TODO
\end{abstract}

\begin{IEEEkeywords}
component, formatting, style, styling, insert
\end{IEEEkeywords}

\section{Introduction -- Simon}
asdfas
\\
\\
\\
\\
\\
\\
\\
\\
\\
\\
\\
\\
\\
\\
\\
\\
\\
\\
\\
\\
\\
\\
\\
\\
\\
\\
\\
\\
\\
\\
\\
\\
\\
\\
\\
\\
\\
\\
\\
\\
\\
\\
\\
\\
\\
\\
\\
\\
\\
\\
\\
\\
\\
\\
\\
\\
intro


\section{Related Work -- Paul}
Refer to articles, how they collected data
did they use mobile phones or separate acceleration sensor devices
body placement of the sensor
The experimental results show that when the sensor is placed on different rigid body, different models are required for certain activities ~\cite{Henpraserttae2011}.
\subsection{Subsection blabla}

subsection

\section{Evaluation -- Simon}

\subsection{Results}
TODO add image
\\
\\
\\
\\
\\
\\
\\
\\
\\
\\
\\
\\
\\
\\
\\
\\
\\
\\
\\
\\
\\
\\
\\
\\
\\
\\
\\
\\
\\
\\
\\
\\
\\
\\
\\
\\
\\
\\
\\
\\
\\
\\
\\
\\
\\
\\
\\
\\
\\
\\
\\
\\
\\
\\
\\
\\
\\
\\
\\
\\
\\
\\
\\
\\
\\
\\
\\
\\
\\
\\
\\
\\
\\
\\
\\
\\
\\
\\
\\
\\
\\
\\
\\
\\
\\
\\
\\
\\
\\
\\
\\
\\
\\
\\
\\
\\
\\
\\
\\
\\
\\
\\
\\
\\
\\
\\
\\
\\
\\
\\
TODO Charts oder shit

\subsection{Analysis}

TODO some analysis shit

\section{Conclusion -- Paul}

TODO Conclude or smth

\section*{List of Abbreviations}
\begin{acronym}[XXXXXXXX]
 \acro{HAR}{Human Activity Recognition}
\end{acronym}
ias
\\
\\
\\
\\
\\
\\
\\
\\
\\
\\
\\
\\
\\
\\
\\
\\
\\
\\

TODO Abbreviations

\addcontentsline{toc}{section}{References}
\bibliographystyle{unsrt}
\bibliography{references}
asdfiajsd
%%%----------------------------------------------------------
%\MakeBibliography                        				% references
%%%----------------------------------------------------------
%\printbibliography
%\bibliography{references}  	% name of bibliography file (references.bib)
%\bibliographystyle{ieeetr}
\\
\\
\\
\\
\\
\\
\\
\\
\\
\\
\\
\\
\\
\\
\\
\\
\\
\\
\\
\\
\\
\\
\\
\\
TODO Refs



\end{document}
